\documentclass[11pt, a4paper, sans]{moderncv}
\usepackage[utf8]{inputenc}
%\usepackage[scale=0.8, top=2cm, bottom=2cm]{geometry}
\usepackage{multicol}
\usepackage{graphbox}
\usepackage{enumitem}
\usepackage{amssymb}
\usepackage{anysize}
\usepackage{anyfontsize}
\marginsize{1.5cm}{1.5cm}{0.5cm}{0.4cm}

\moderncvstyle{casual}
\moderncvcolor{black}
\setlength{\hintscolumnwidth}{3.5cm} %taille de la bande gauche

\name{Jérémy}{Zangla} 


\begin{document}
\begin{flushright}
\fontfamily{lmss}\fontsize{27pt}{1}\selectfont{\textrm{\color{gray}Jérémy \color{black}ZANGLA\\}}
\normalsize
\color{black}
Lebenslauf\\

\end{flushright}
\rule{17cm}{0.01pt}

\footnotesize
\faHome~: 22 rue Saint-Martial, 63730 Les Martres-de-Veyre, Frankreich \hfill Führerscheinklasse B : \faCar\\
\faEnvelopeSquare~: zangla.jeremy@gmail.com \hfill \href{https://github.com/gagoi}{gagoi : \faGithubSquare} \\
\faPhoneSquare~: +336 06 49 76 29 \hfill \href{https://www.linkedin.com/in/j\%C3\%A9r\%C3\%A9my-zangla-a18336137/}{Jérémy Zangla : \faLinkedinSquare}
\normalsize
\rule{17cm}{0.1pt}
\section{Persönliches}
Nationalität: Französisch
Geburtsdatum: 05.03.1998\\
Geburtdort: Clermont-Ferrand, Frankreich\\
Familienstand: ledig
\section{Ausbildung}
\cventry{09/2018 -- heute \footnotesize(absolvierte im Jahr 2021)\normalsize}{Abschluss als Computeringenieur}{ISIMA}{Clermont-Ferrand}{}{
	Im zweiten Studienjahr, in der zweite studiengang : \textit{Software Engineering und Computersysteme}.
	\vspace{-1.5mm}
	\begin{itemize}[label=\textbullet]
		\item Projekt in Paaren (2019) : Herstellung eines Sumoroboters (C++, Mbed, Nucleo).
	\end{itemize}
}
\cventry{09/2016 -- 06/2018}{Prep'ISIMA}{UCA \& ISIMA}{Clermont-Ferrand}{}{
	\begin{itemize}[label=\textbullet]
		\item Projekt in Paaren (2018) : Herstellung eines Linienfolgeroboters (C++, Arduino, Bluetooth).
	\end{itemize}
}
\cventry{09/2013 -- 07/2016}{Wissenschaftliches Abitur}{Lycée Lafayette}{Clermont-Ferrand}{}{
	\begin{itemize}[label=\textbullet]
		\item Projekt in Paaren (2015-2016) : Entwicklung einer JAVA-Bibliothek zur Erstellung von 2D-Spielen.
	\end{itemize}}


\section{Erfahrungen}
\cventry{2015 -- Heute}{Familienunternehmen (Spargelproduktion)}{Les Martres-de-Veyre}{}{}{
	\begin{itemize}[label=\textbullet]
		\setlength\itemsep{0mm}
		\item Management von Arbeiten unter schwierigen Bedingungen (Wetter, Verfügbarkeit von Personal).
		\item Identifizierung und Behebung aufgetretener Probleme.
		\item Teilnahme an landwirtschaftlichen Arbeiten und Verkauf der Produktion an den Markt.
	\end{itemize}
}
\cventry{03/2018 -- Heute}{Schatzmeister}{Isibot (Robotikclub auf ISIMA)}{}{}{
	\begin{itemize}[label=\textbullet]
		\setlength\itemsep{0mm}
		\item Erstellung eines Projektplans und Verhandlung von Stipendien mit dem Studentensekretariat.
		\item Teilnahme an Kursen für Anfänger in der Robotik.
	\end{itemize}
}

\section{Weitere Qualifikationen}
\subsection{Sprachkenntnisse}
\cvlanguage{Französisch}{Muttersprache}{}
\cvlanguage{Englisch}{Gut}{TOIC 950}
\cvlanguage{Deutsch}{Grundkenntnissebutant}{}
\subsection{EDV-Kenntnisse}
	\begin{itemize}[label=\textbullet]
		\setlength\itemsep{0mm}
		\item Computer Sprache : Java, C, C++, Python, Scheme, JavaFX, Arduino, MBed, SFML.
		\item IDEs : Eclipse, Android Studio, Arduino IDE, Mbed platform, Visual Studio (et VSCode).
		\item Büro-Tools : Microsoft Office (Word, Excel, PowerPoint), LibreOffice (Writer, Calc).
		\item Andere : LaTex, SQLite, MySQL, JSON, XML, Git, Valgrind, Catch2 Framework.
	\end{itemize}

\section{Sonstige Aktivitäten}
\begin{itemize}[label=\textbullet]
	\setlength\itemsep{0mm}
	\item Teilnahme an 6 Game Jams (Entwicklung von Videospielen in begrenzter Zeit von 24h bis 72h).
	\item Handi'Tutorat (Schulförderung für Gymnasiasten mit Behinderungen).
\end{itemize}

\bigskip
\raggedleft Dienstag, den einunddreißigten Oktober 2019, in Clermont-Ferrand

\end{document}